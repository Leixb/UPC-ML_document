%! TEX root = **/000-main.tex
% vim: spell spelllang=en:

\section{Introduction}%
\label{sec:introduction}

The NBA Draft market is a highly competitive market where teams bid and draft
players in the hopes of getting the next star player. To that end, team
recruiters are always on the lookout for new talent appearing in collegiate
leagues. Finding new talent before other teams can give the edge over the
competition. There are numerous websites and resources available that compile
statistics of players, as such, modelling the performance of players using
statistical techniques is common practice.

\subsection{Goals}%
\label{sub:goals}

The aim of this project is to analyze to what extent the performance of a player
in collegiate basketball leagues can be used to predict the player's eventual
performance in the NBA.

\subsection{Data sources}%
\label{sub:data-sources}

We used data from the following sources:

\begin{itemize}
  \item \href{http://www.basketball-reference.com}{basketball-reference.com} \cite{noauthor_basketball_nodate}
  \item \href{http://www.nbastuffer.com}{nbastuffer.com} \cite{noauthor_nbastuffer_2017}
  \item NBA official statistics \cite{noauthor_draft_nodate}
  \item Kaggle collegiate and NBA player dataset \cite{noauthor_college_nodate,noauthor_t-rank_nodate}
\end{itemize}

The main source was basketball-reference. We used the rest to add some
additional variables.

\subsubsection{Data collection}%
\label{ssub:data-collection}

We collected the data using a simple web scraper written in Python, which can be
found in our source code repository. It basically gets all players drafted between the years 2000 and 2020. For each of them, it obtains statistics about their NBA games and all the information available about their statistics before they entered the league.

\section{Related previous work}%
\label{sec:previous-work}

There are a lot of studies that aim to model the performance of NBA players
or basketball players in general \cite{terner_modeling_2020,
casals_modelling_2013, kokanauskas_modelling_2021}.  But there are not as many
studies on the collegiate league performance related to national league.

There have been several studies on the performance of players in college
basketball and the national league. However, most of them focus on the so called
\emph{one-year-and-done} rule \cite{noauthor_nba_2006} set by the NBA in 2006.
This rule requires players to be 19 years old and have finished high-school at least
1 year prior to being drafted in the NBA. In practice, this means that most
young talent ends up playing 1 year in the college league and then drops out to
go to the national league.

These studies have found that players with less college experience have slightly
better offensive statistics in their first year on the national league compared to their
college graduates counterparts \cite{ashley_explaining_2017}. Despite their better statistics,
they commit more mistakes such as fouls and turnovers \cite{zestcott_one_2020}.

